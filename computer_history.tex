%%%%%%%%%%%%%%%%%%%%%%%%%%%%%%%%%%%%%%%%%%%%%%%%%%%%%%%%%%%%%%%%%%%%%%%%%%%%%%%%

% IEEEconf.cls file must exist in the same directory as the TeX file you want to compile
\documentclass[letterpaper, 10 pt, conference]{IEEEconf}

\title{\LARGE \bf
COMPUTER HISTORY\\
\large Group Topic Expressed In A Few Words
}

\author{Group Number\\
\small Group Member 1\\
\small Group Member 2\\
\small Group Member 3\\
}

% Image/graphics support
\usepackage{graphicx}
\graphicspath{ {./images/} }

% Formatting for lists
\usepackage{enumitem}

% Formatting for media
\usepackage{float}
\restylefloat{table}
\restylefloat{figure}

\begin{document}


\maketitle
\thispagestyle{empty}
\pagestyle{empty}


%%%%%%%%%%%%%%%%%%%%%%%%%%%%%%%%%%%%%%%%%%%%%%%%%%%%%%%%%%%%%%%%%%%%%%%%%%%%%%%%
\section*{ABSTRACT}
\textit{
This document is a basic template for writing conference-style
reports in LaTeX. You will use this template when writing
your report; you will need to replace all text (excluding
section headers or preamble information) with the content
of your report.
}

%%%%%%%%%%%%%%%%%%%%%%%%%%%%%%%%%%%%%%%%%%%%%%%%%%%%%%%%%%%%%%%%%%%%%%%%%%%%%%%%
\section{INTRODUCTION}

In a modern world of complex operation modes like asynchronous and functional programming, 
It is important to appreciate the simple methodology of real-time programming. It's a concept 
that emerged alongside what we consider modern computing and is still present in some form in nearly 
every computing application today. Everything from anti-lock brake controls, to space flight, to nuclear powerplant control units; 
everything uses real-time programming. The idea is simple: every input produces a near-instant output, and tasks are scheduled and 
run at exact times. Any delay, even a few milliseconds can be considered as a failure, and thus while it is a simple concept on paper, 
it is far more difficult to achieve in practice. This struggle and development is ongoing and continues to produce new solutions, both hardware and 
software, every year.

%%%%%%%%%%%%%%%%%%%%%%%%%%%%%%%%%%%%%%%%%%%%%%%%%%%%%%%%%%%%%%%%%%%%%%%%%%%%%%%%
\section{TIME PERIOD}

You should describe the time period in which your topic was
invented or used here. Also include the context for why your
topic was created or for how it is used. Any specific historical
information should be included here.

%%%%%%%%%%%%%%%%%%%%%%%%%%%%%%%%%%%%%%%%%%%%%%%%%%%%%%%%%%%%%%%%%%%%%%%%%%%%%%%%
\section{COMPUTER HARDWARE}
The hardware in real-time computing must be extremely dependable and precise.
 This is critical because any delay may cause the entire system to fail.
  Real-time Computing is used in many real life situations, from car breaks to launching missiles. 
  A couple examples of the hardware in real-time computing is Xilinx FPGA and
   SoC boards, System-on-Modules, and Alveo Data Center accelerator cards.
    Another way to create more powerful hardware is by adding a real-time micro kernel 
    between the normal hardware and the Linux kernel. 

\section{COMPUTER SOFTWARE}

Describe the software used for your chosen topic if any,
and state any uses of the software that your topic had.
If your topic does not have or use software, describe why it
doesn't use software and how it functions without it.

\section{CONCLUSION}

Conclude your research paper with any reflections on what you
learned about your topic. Was this what you expected to find?
Did you find any facts that surprised you? You may add other
personal reflections about the topic here.

\section*{REFERENCES}

Below are basic formats for different types of references.

\begin{enumerate}[label={[\arabic*]}]
  \item “What Is Real-Time Computing?” RTXI, http://rtxi.org/docs/tutorials/2014/12/06/what-is-real-time-computing/.

\end{enumerate}

\end{document}

