%%%%%%%%%%%%%%%%%%%%%%%%%%%%%%%%%%%%%%%%%%%%%%%%%%%%%%%%%%%%%%%%%%%%%%%%%%%%%%%%

% IEEEconf.cls file must exist in the same directory as the TeX file you want to compile
\documentclass[letterpaper, 10 pt, conference]{IEEEconf}

\title{\LARGE \bf
COMPUTER HISTORY\\
\large The History of Real-Time Computing
}

\author{Group 5\\
\small Nash Taylor\\
\small Summer Bronson\\
\small Thomas Merl\\
\small Elijah Pelofske
}

% Image/graphics support
\usepackage{graphicx}
\graphicspath{ {./images/} }

% Formatting for lists
\usepackage{enumitem}

% Formatting for media
\usepackage{float}
\restylefloat{table}
\restylefloat{figure}

% Multiline commenting
\usepackage{verbatim}

\begin{document}


\maketitle
\thispagestyle{empty}
\pagestyle{empty}


%%%%%%%%%%%%%%%%%%%%%%%%%%%%%%%%%%%%%%%%%%%%%%%%%%%%%%%%%%%%%%%%%%%%%%%%%%%%%%%%
\section{INTRODUCTION}

\begin{comment}
This document is an example of Assignment 3 in CSE/IT101.
You should describe what topic you chose and
why you chose that topic. Also provide a summary of the
topic, in a general sense. This section should describe any
background information that the reader needs to know to
understand your topic.
\end{comment}

%%%%%%%%%%%%%%%%%%%%%%%%%%%%%%%%%%%%%%%%%%%%%%%%%%%%%%%%%%%%%%%%%%%%%%%%%%%%%%%%
\section{TIME PERIOD}

\begin{comment}
You should describe the time period in which your topic was
invented or used here. Also include the context for why your
topic was created or for how it is used. Any specific historical
information should be included here.
\end{comment}
%%%%%%%%%%%%%%%%%%%%%%%%%%%%%%%%%%%%%%%%%%%%%%%%%%%%%%%%%%%%%%%%%%%%%%%%%%%%%%%%
\section{COMPUTER HARDWARE}

\begin{comment}
You should list the specification of any hardware your topic
uses here. If you want make a table here, please label the table
and include discussion on what components are included in the
table and why. See Table
\ref{tbl:example} for an example of a table.
The labels/captions of the table should be put at the bottom
of the table.
\end{comment}

\begin{table}[h!]
\begin{center}
\begin{tabular}{||c | c | c | c||} 
\hline
  & Col1 & Col2 & Col3 \\ [0.5ex]
\hline\hline
Row1 & (1,1) & (1,2) & (1,3) \\ 
\hline
Row2 & (2,1) & (2,2) & (2,3) \\
\hline
Row3 & (3,1) & (3,2) & (3,3) \\
\hline
\end{tabular}
\caption{Example of a table}
\label{tbl:example}
\end{center}
\end{table}

\begin{comment}
You might want to put figures in the document. Please
remember to label them. The labels/captions of the figures
should be put at the bottom of the figure. See Figure
\ref{fig:example} for an example of how to use figures.
You will need to place the figure in an \texttt{images/} folder
in your working directory.
\end{comment}

\begin{figure}[h!]
\centering
\includegraphics[width=0.5\textwidth]{spiral.png}
\caption{Example of a figure}
\label{fig:example}
\end{figure} 

%%%%%%%%%%%%%%%%%%%%%%%%%%%%%%%%%%%%%%%%%%%%%%%%%%%%%%%%%%%%%%%%%%%%%%
\section{COMPUTER SOFTWARE}

\begin{comment}
Describe the software used for your chosen topic if any,
and state any uses of the software that your topic had.
If your topic does not have or use software, describe why it
doesn't use software and how it functions without it.
\end{comment}

The main piece of software on a real-time computer is a real-time operating system (RTOS). An RTOS is an operating system
that processes data as it comes into the system with little to no delay. The RTOS has a very limited amount of time (usually
measured in tenths of a second) to process the data and if it can't process the data in time, the entire system will fail. While this seems useless, it actually
is very important. Real time operating systems can be found in computer systems for cars and spacecraft for NASA, where performance
is key. For example, the computer system controlling the self driving function in a Tesla needs to process data in near real-time 
in order to prevent a crash.
%%%%%%%%%%%%%%%%%%%%%%%%%%%%%%%%%%%%%%%%%%%%%%%%%%%%%%%%%%%%%%%%%%%%%%
\section{CONCLUSION}

\begin{comment}
Conclude your research paper with any reflections on what you
learned about your topic. Was this what you expected to find?
Did you find any facts that surprised you? You may add other
personal reflections about the topic here.
\end{comment}

%%%%%%%%%%%%%%%%%%%%%%%%%%%%%%%%%%%%%%%%%%%%%%%%%%%%%%%%%%%%%%%%%%%%%%%
\section*{REFERENCES}




\begin{enumerate}[label={[\arabic*]}]
\item ``Real-time Operating System.'' Wikipedia, Wikimedia Foundation, 26 Sept. 2019, \\ en.wikipedia.org/wiki/Real-time\_operating\_system.
\item “What Are the Five Most Commonly Used Real-Time Operating Systems?” Stack Overflow, stackoverflow.com/questions/5281848/what-are-the-five-most-commonly-used-real-time-operating-systems.
\end{enumerate}

\end{document}

