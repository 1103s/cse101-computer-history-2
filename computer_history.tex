%%%%%%%%%%%%%%%%%%%%%%%%%%%%%%%%%%%%%%%%%%%%%%%%%%%%%%%%%%%%%%%%%%%%%%%%%%%%%%%%

% IEEEconf.cls file must exist in the same directory as the TeX file you want to compile
\documentclass[letterpaper, 10 pt, conference]{IEEEconf}

\title{\LARGE \bf
COMPUTER HISTORY\\
\large Group Topic Expressed In A Few Words
}

\author{Group Number\\
\small Group Member 1\\
\small Group Member 2\\
\small Group Member 3\\
}

% Image/graphics support
\usepackage{graphicx}
\graphicspath{ {./images/} }

% Formatting for lists
\usepackage{enumitem}

% Formatting for media
\usepackage{float}
\restylefloat{table}
\restylefloat{figure}

\begin{document}


\maketitle
\thispagestyle{empty}
\pagestyle{empty}


%%%%%%%%%%%%%%%%%%%%%%%%%%%%%%%%%%%%%%%%%%%%%%%%%%%%%%%%%%%%%%%%%%%%%%%%%%%%%%%%
\section*{ABSTRACT}
\textit{
This document is a basic template for writing conference-style
reports in LaTeX. You will use this template when writing
your report; you will need to replace all text (excluding
section headers or preamble information) with the content
of your report.
}

%%%%%%%%%%%%%%%%%%%%%%%%%%%%%%%%%%%%%%%%%%%%%%%%%%%%%%%%%%%%%%%%%%%%%%%%%%%%%%%%
\section{INTRODUCTION}

In a modern world of complex operation modes like asynchronous and functional programming, 
It is important to appreciate the simple methodology of real-time programming. It's a concept 
that emerged alongside what we consider modern computing and is still present in some form in nearly 
every computing application today. Everything from anti-lock brake controls, to space flight, to nuclear powerplant control units; 
everything uses real-time programming. The idea is simple: every input produces a near-instant output, and tasks are scheduled and 
run at exact times. Any delay, even a few milliseconds can be considered as a failure, and thus while it is a simple concept on paper, 
it is far more difficult to achieve in practice. This struggle and development is ongoing and continues to produce new solutions, both hardware and 
software, every year.

%%%%%%%%%%%%%%%%%%%%%%%%%%%%%%%%%%%%%%%%%%%%%%%%%%%%%%%%%%%%%%%%%%%%%%%%%%%%%%%%
\section{TIME PERIOD}

Real time computing was developed alongside standard use cases for computers when real time simulations were implemented. In 
some early systems starting in the 1970's real time operating systems could be used on personal desktop computers. However, 
real time computing systems primary use case is in highly precise and time critical real world systems such with cars and 
automated military systems and simulations. Therefore, they subsequently gained the attention of industry and government 
usage across a massive array of applications. Although they are an incredibly specific use case, where in general timing 
accuracy and task management, real time computing systems are still in use today in many modern digital systems.

%%%%%%%%%%%%%%%%%%%%%%%%%%%%%%%%%%%%%%%%%%%%%%%%%%%%%%%%%%%%%%%%%%%%%%%%%%%%%%%%
\section{COMPUTER HARDWARE}
The hardware in real-time computing must be extremely dependable and precise.
 This is critical because any delay may cause the entire system to fail.
  Real-time Computing is used in many real life situations, from car breaks to launching missiles. 
  A couple examples of the hardware in real-time computing is Xilinx FPGA and
   SoC boards, System-on-Modules, and Alveo Data Center accelerator cards.
    Another way to create more powerful hardware is by adding a real-time micro kernel 
    between the normal hardware and the Linux kernel. 

\section{COMPUTER SOFTWARE}

The main piece of software on a real-time computer is a real-time operating system (RTOS). An RTOS is an operating system
that processes data as it comes into the system with little to no delay. The RTOS has a very limited amount of time (usually
measured in tenths of a second) to process the data and if it can't process the data in time, the entire system will fail. While this seems useless, it actually
is very important. Real time operating systems can be found in computer systems for cars and spacecraft for NASA, where performance
is key. For example, the computer system controlling the self driving function in a Tesla needs to process data in near real-time
in order to prevent a crash.

\section{CONCLUSION}
In conclusion, real time computing is an extremely niche and incredibly difficult to develop and implement type of computing. Real time computing is often broken into three different catagories of acceptable time tolerances. However, even for the less constrained real time computing systems, the interplay between critical scheduling of programs and processes is incredibly precise, often requiring precision down to the microsecond. The reason for this is largely the application regime for real time computing, which includes life and death consequences for even miniscule errors. Such critical systems include braking systems in cars and missile guidance systems. Although incredibly specific in its potential use cases, real time computing is an integral part of modern real world applications.

\section*{REFERENCES}

Below are basic formats for different types of references.

\begin{enumerate}[label={[\arabic*]}]
  \item “What Is Real-Time Computing?” RTXI, http://rtxi.org/docs/tutorials/2014/12/06/what-is-real-time-computing/.
  \item ``Real-time Operating System.'' Wikipedia, Wikimedia Foundation, 26 Sept. 2019, \\ en.wikipedia.org/wiki/Real-time\_operating\_system.
  \item “What Are the Five Most Commonly Used Real-Time Operating Systems?” Stack Overflow, stackoverflow.com/questions/5281848/what-are-the-five-most-commonly-used-real-time-operating-systems. 
  \item Kang Shin et al. Real-Time Computing: A new Discipline of Computer Science and Engineering.
\end{enumerate}

\end{document}

